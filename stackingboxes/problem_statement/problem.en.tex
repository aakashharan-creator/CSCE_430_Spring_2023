\problemname{Babylonian Boxes}

The people of 2D Babylon have the brilliant idea to build a tower to reach heaven. 
At their disposal they have rectangles with varying widths and heights. In order to construct the 
tower, they can stack these rectangles. The only problem is they can stack a rectangle on another 
only if it's base is lesser than or equal to the rectangle directly below it. Due to limits on construction infrastructure,
the people of Babylon cannot change the relative order of the boxes they see. It must be taken in the sequence they 
see it. For example, a rectangle that appears earlier in the input cannot be placed above a rectangle 
that appears later in the sequence even if it obeys the base rule. One thing they can do however, is rotate the rectangle 
to use the height or width as a base as long as the base stacking rule isn't violated. Given a set of rectangles 
help the people of 2D Babylon build the tallest tower they can. There's no way this is a bad idea, right?

\section*{Input}

The input consists of one test case. The first line contains an integer $n$ $(1 \le n \le 1000)$, the number of rectangles. 
The following $n$ lines contains two space separated integers, $a$ and $b$ $(1 \le a, b \le 200)$, the dimensions
of the $ith$ rectangle.

\section*{Output}

Output a single integer, the tallest tower that can be built.